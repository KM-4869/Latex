\documentclass{ctexart}
\usepackage{graphicx}
\usepackage{amsmath}
\usepackage[colorlinks=true,allcolors=red]{hyperref}%插入代码也需要此行
\usepackage{bm}
\usepackage{float}
\usepackage{enumerate}
\usepackage{listings}%插入关键字高亮的代码
\lstset{basicstyle=\small,
basicstyle=\tt,
keywordstyle=\color{blue},
commentstyle=\color{green},
frame=shadowbox,
numbers=left,
numberstyle=\tiny\color{red},
columns=flexible,}
\begin{document}
\begin{center}
\thispagestyle{empty}
\Large\textbf{应用KALMAN滤波估计落体运动}
\end{center}
\vfill
\begin{table}[h]
    \centering
    \large
    \begin{tabular}{cc}
    \textbf{课程名称:} & 组合导航 \\ \cline{2-2}
    \textbf{班级:} & 导航工程3班 \\ \cline{2-2}
    \textbf{姓名:} & 叶通 \\  \cline{2-2}
    \textbf{学号:} & 2020302142138 \\ \cline{2-2}
    \textbf{时间:} & \today \\ \cline{2-2}
    \end{tabular}
\end{table}

\newpage
\large\textbf{问题描述:假设有一物体从坐标原点0处,受到加速度为$g=10m/s^2$,以初速度为0做落体运动,运动过程中受到阻力为$f=k\cdot v$,其中阻力系数$k=0.1$,假设原点处有一测速仪每秒输出落体的速度,利用测速仪和此运动模型估计落体的速度和位置}

\vspace{2cm}
设物体的二维状态为位置和速度$\left[\begin{array}{c}r(t)\\v(t)\end{array}\right]$,物体的运动状态可以描述为
\begin{equation*}
\begin{bmatrix}
\dot{r}(t)\\
\dot{v}(t)\end{bmatrix}=\begin{bmatrix}
0&1\\
0&-0.1
\end{bmatrix}\begin{bmatrix}
r(t)\\
v(t)
\end{bmatrix}+\begin{bmatrix}
0\\
1
\end{bmatrix}g+\begin{bmatrix}
0\\
1
\end{bmatrix}e(t)
\end{equation*}

观测方程可以描述为
\begin{equation*}
Z(t)=\begin{bmatrix}
0&1
\end{bmatrix}\begin{bmatrix}
r(t)\\
v(t)
\end{bmatrix}+\Delta(t)
\end{equation*}
\vspace{1cm}
设初始状态的协方差阵$\bm{P}$为$\left[\begin{array}{cc}
5&0\\
0&5
\end{array}\right]$,过程噪声矩阵\textbf{Q}=$\left[\begin{array}{cc}
1&1\\
1&1
\end{array}\right]$,观测噪声矩阵\textbf{R}为[4]。求解微分方程可以解得状态转移矩阵
\begin{equation*}
\bm{\varPhi}=e^{\bm{A}t}=\begin{bmatrix}
  1&0.95162581\\
  0&0.90483741
\end{bmatrix}
\end{equation*}
再以此求解输入增益矩阵\textbf{B}。

\vspace{2cm}
根据运动学方程可以写出
\begin{equation*}
\dfrac{dv}{dt}=10-0.1v
\end{equation*}
移项后对两边同时积分后得
\begin{equation*}
-10ln(10-0.1v)+10ln(10)=t
\end{equation*}

以此为真实的物体运动速度随时间的变化规律,每秒取离散的时间,计算1-50秒的速度作为物体运动速度的真实值,用Matlab函数将真实值上加上正态分布随机数作为数据观测值序列。再将上述定义好的变量和观测值调用给出的Matlab开源函数实现每秒对落体速度的估计。部分代码示意如下。
\begin{lstlisting}[language=Matlab]
for i=1:1:50
    [X_pred,P_pred]=kf_predict(X,P,A,Q,B,U);
    [X,P,K,IM,IS,LH]=kf_update(X_pred,P_pred,Y(i),H,R);
end
\end{lstlisting}
\newpage
将每次输出结果保存,画出运动速度的真值曲线,观测曲线,卡尔曼滤波平滑后曲线有
\begin{figure}[H]
\includegraphics[height=10cm,width=1.0\linewidth]{落体运动速度曲线.jpg}
\caption{落体运动速度曲线}
\end{figure}

可以看到经过Kalman滤波平滑后的速度处于理论值与观测值之间,对观测值进行了修正。
\newpage
作出速度的方差曲线如下
\begin{figure}[H]
\includegraphics[height=10cm,width=1.0\linewidth]{速度方差曲线.jpg}
\caption{速度方差曲线}
\end{figure}

速度的方差可以迅速收敛为较小的值,对速度的估计比较准确。
\newpage
作出位置方差曲线如下
\begin{figure}[H]
\includegraphics[height=10cm,width=1.0\linewidth]{位置方差曲线.jpg}
\caption{位置方差曲线}
\end{figure}

发现位置的方差并没有收敛,而是呈随时间线性增大。这里经过多次调整$\bm{Q}$阵,$\bm{P}$阵和$\bm{R}$阵,作图后发现只是其增大的速度有所改变,结果依然未收敛,最后也只能对未收敛的原因作一些猜测。
\begin{enumerate}[(1)]
\item $\bm{Q}$阵依旧不太合理。由于这里$\bm{Q}$阵并没有按正常步骤由状态转移矩阵和噪声矩阵等经过积分求出,而是直接给出的,会与理想的问题假设有所差异。
\item 由问题假设本身决定。由于位置是由于上一时刻位置和速度二者共同决定,故方差也受二者的影响,相当于二者的叠加,导致难以收敛。
\item 由于问题假设中只对状态量之一的速度进行了观测,所以缺少观测值约束的位置量会一直发散。
\end{enumerate}
\end{document}