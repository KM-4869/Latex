\documentclass{ctexart}
\usepackage{graphicx}
\usepackage[colorlinks=true,allcolors=red]{hyperref}
\usepackage{enumerate}
\usepackage{amsmath}
\usepackage{float}

\usepackage{listings}%插入关键字高亮的代码
\lstset{basicstyle=\small,
basicstyle=\tt,
keywordstyle=\color{blue},
commentstyle=\color{green},
frame=shadowbox,
numbers=left,
numberstyle=\tiny\color{red},
columns=flexible,}
\begin{document}
\newpage
\thispagestyle{empty}
    \begin{center}
        \Huge\textbf{室内定位课程作业报告}
    \end{center}
\vspace*{\fill}

\begin{table}[h]
    \centering
    \large
    \begin{tabular}{cc}
    \textbf{课程名称:} & 室内定位技术 \\ \cline{2-2}
    \textbf{班级:} & 导航工程3班 \\ \cline{2-2}
    \textbf{姓名:} & 叶通 \\  \cline{2-2}
    \textbf{学号:} & 2020302142138 \\ \cline{2-2}
    \textbf{时间:} & \today \\ \cline{2-2}
    \end{tabular}
\end{table}

\newpage
\pagenumbering{Roman}
\setcounter{page}{1}
\tableofcontents

\newpage
\setcounter{page}{1}
\pagenumbering{arabic}

\section{几何定位}
\subsection{实现步骤}
使用距离交会的方式估计无人机的位置,测量值为无人机到四个基站的距离,基站坐标已知。

设基站坐标为$X_S,Y_S$,所估计的无人机坐标为X,Y,几何距离为d,可列观测方程
\begin{equation}
d=\sqrt{(X_S-X)^2+(Y_S-Y)^2}
\end{equation}
设无人机近似坐标为$X_0,Y_0$,在此处经泰勒展开线性化后有
\begin{equation}
d=d_0+\frac{X_0-X_S}{d_0}dX+\frac{Y_0-Y_S}{d_0}dY
\label{eq:2}
\end{equation}
其中$d_0=\sqrt{(X_S-X_0)^2+(Y_S-Y_0)^2},dX=(X-X_0),dY=(Y-Y_0)$

此过程选择使用C++来实现。先定义四个基站坐标,并以四个坐标的均值作为无人机位置的初值。之后使用循环语句,按行读入四个距离观测值,按\eqref{eq:2}构造B矩阵和l矩阵,使用事先写好的最小二乘函数迭代求解。当小于阈值时退出,继续解算下一个位置,直到文件结束。

\subsection{结果展示}
\begin{figure}[H]
\includegraphics[height=7cm,width=0.9\linewidth]{无人机轨迹图.jpg}
\caption{无人机轨迹图}
\end{figure}

\section{行人航迹推算}
\subsection{实现步骤}

此过程采用Matlab实现。
\begin{enumerate}[(1)]
\item 导入传感器数据,使用plot函数作出传感器数据图。
\item 使用加速度计数据计算水平姿态角。
\begin{lstlisting}[language=Matlab]
%计算水平姿态角
roll=rad2deg(atan2(-y_acc,-z_acc));
pitch=rad2deg(atan2(x_acc,sqrt(y_acc.^2+z_acc.^2)));
\end{lstlisting}
注意需要将弧度转角度。以及向量后加“.”表示里面单个元素的运算。

\item 对水平姿态角进行1秒钟的窗口平滑并且作图。
\item 使用磁强计计算航向角
\begin{lstlisting}[language=Matlab]
%磁强计计算航向角
yaw_m=14.11+rad2deg(-atan2(y_mag,x_mag));
\end{lstlisting}
\item 使用陀螺仪计算航向角。由于实验数据为在Z轴竖直于地面的情况下采集的,故航向角的计算可以简化为$yaw_k=yaw_{k-1}+\Delta t\cdot \theta_{k-1}$,其中$\theta$为Z轴陀螺仪在k-1时刻的输出。初始航向为-90°。还需注意将所得结果化为-180°到180°的区间内。
\item 使用三轴加速度计的模来探测脚步。采用峰值探测函数findpeaks,设置最小峰值,找到所有脚步并作图标记。
\item 分别使用磁强计和陀螺仪获取的航向,使用探测出的脚步,进行行人航迹推算。按步长为0.7m,初始位置【0,0】。并作出轨迹图。
\item 对z轴陀螺数据加入一个常值零偏补偿,调整基于陀螺的PDR结果。这里加入的零偏补偿为+0.2°。
\end{enumerate}
\subsection{结果展示}

\begin{figure}[H]
\includegraphics[height=10cm,width=1\linewidth]{传感器数据.jpg}
\caption{传感器数据}
\end{figure}
\begin{figure}[H]
\includegraphics[height=10cm,width=1\linewidth]{水平姿态角.jpg}
\caption{水平姿态角}
\end{figure}
\begin{figure}[H]
\includegraphics[height=10cm,width=1\linewidth]{平滑后水平姿态角.jpg}
\caption{平滑后水平姿态角}
\end{figure}
\begin{figure}[H]
\includegraphics[height=10cm,width=1\linewidth]{解算航向角.jpg}
\caption{解算航向角}
\end{figure}
\begin{figure}[H]
\includegraphics[height=10cm,width=1\linewidth]{脚步探测.jpg}
\caption{脚步探测}
\end{figure}
\begin{figure}[H]
\includegraphics[height=10cm,width=1\linewidth]{PDR解算轨迹.jpg}
\caption{PDR解算轨迹}
\end{figure}
\begin{figure}[H]
\includegraphics[height=10cm,width=1\linewidth]{解算航向角(陀螺Z轴补偿).jpg}
\caption{解算航向角(陀螺Z轴补偿)}
\end{figure}
\begin{figure}[H]
\includegraphics[height=10cm,width=1\linewidth]{PDR解算轨迹(陀螺Z轴补偿).jpg}
\caption{PDR解算轨迹(陀螺Z轴补偿)}
\end{figure}

\section{信号强度指纹识别}
\subsection{实现步骤}

此过程使用Matlab实现。

先读入数据库文件,分割提取出97个参考点的数据信息,将每个参考点的30个信号特征求取均值,最后生成一个$97\times30$的矩阵,作为特征匹配的数据库。

读取测试文件,将每一时刻的信号强度向量分别与97个参考点上的信号强度进行比较,计算二者之间的距离。这里为了降低粗差(出现丢数-110)的影响,使用向量差的1范数(即所有维数的绝对值求和)作为二者之间的距离度量。之后使用sort所有距离进行升序排列并返回一组元素在原向量的位置的索引值,取出前三个距离最短的参考点。用距离的倒数作为权重,加权平均求取每一测试点的位置。
\subsection{结果展示}

\begin{figure}[H]
\includegraphics[height=10cm,width=1\linewidth]{数据库匹配结果图.jpg}
\caption{数据库匹配结果图}
\end{figure}





\end{document}