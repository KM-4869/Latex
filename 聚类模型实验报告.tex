\documentclass{ctexart}
\usepackage{graphicx}
\usepackage[colorlinks=true,allcolors=red]{hyperref}
\usepackage{enumerate}
\usepackage{amsmath}
\usepackage{float}
\usepackage{tabularray}%表格
\usepackage{listings}%插入关键字高亮的代码
\usepackage{nicematrix}
\usepackage{multirow}%\multicolumn{3}{c}{\multirow{3}{*}{\large $R_{3\times3}$}}  \multicolumn{要合并的列数}{对齐方式}{内容}  \multirow{要合并的行数}{*默认}{内容}
\usepackage{geometry}
\usepackage{pythonhighlight}
\usepackage{booktabs}


\geometry{a4paper,scale=0.7}

\lstset{basicstyle=\small,
basicstyle=\tt,
keywordstyle=\color{blue},
commentstyle=\color{green},
frame=shadowbox,
numbers=left,
numberstyle=\tiny\color{red},
columns=flexible,}
\begin{document}
\newpage
\thispagestyle{empty}
%\vspace*{\fill}
    \begin{center}
        \Huge\textbf{机器学习聚类模型实验报告}
    \end{center}
\vspace*{\fill}

\begin{table}[h]
    \centering
    \Large
    \begin{tabular}{cc}
    \textbf{学院:} & 测绘学院 \\ \cline{2-2}
    \textbf{专业:} & 导航工程 \\ \cline{2-2}
    \textbf{姓名:} & 叶通 \\  \cline{2-2}
    \textbf{学号:} & 2020302142138 \\ \cline{2-2}
    \textbf{教师:} & 郭迟 \\ \cline{2-2}
    \textbf{时间:} & 2023年1月23日 \\ \cline{2-2}
    \end{tabular}
\end{table}

\newpage
\pagenumbering{Roman}
\setcounter{page}{1}
\tableofcontents

\newpage
\setcounter{page}{1}
\pagenumbering{arabic}

\section{实验要求与内容}
\subsection{实验内容}
掌握基于Python语言,使用PCA方法对样本的降维,实现划分式聚类方法Kmeans和基于密度的聚类方法DBSCAN对样本实现聚类,掌握两种方法的模型设计思想,区别与共性。
\subsection{实验要求}
\begin{enumerate}[(1)]
\item 掌握sklearn库的使用
\item 使用PCA方法实现对样本的降维
\item 对数据集进行可视化
\item 分别使用两种方法对样本完成聚类任务
\item 分别使用外部指标和内部指标对聚类进行评估
\end{enumerate}

\section{编程实现过程}
\subsection{数据集生成}
使用make blobs函数,生成样本个数为500,含有8个特征纬度,中心类别数目为3的数据集,具体代码如下:
\begin{python}
X, y = make_blobs(n_samples=500, n_features=8,  centers=3,  cluster_std=4.0,  random_state=3)
\end{python}

\subsection{特征降维PCA}
使用PCA方法,将刚刚生成的样本降维,新的维数为3维以方便可视化绘图,具体代码如下:
\begin{python}
# PCA方法降维
pca = PCA(n_components=3)
X_rd = pca.fit_transform(X)
\end{python}

\subsection{数据集可视化}
将刚刚降到3为的样本在空间直角坐标系中绘制出来,表示成散点图的形式。
\begin{python}
# 数据集可视化
fig1 = plt.figure(figsize=(10, 8))
ax1 = Axes3D(fig1)
x1 = X_rd[:, 0]
x2 = X_rd[:, 1]
x3 = X_rd[:, 2]
ax1.scatter(x1, x2, x3, s=25, c='c', marker='h')
plt.show()
\end{python}

绘制出的图像如下:
\begin{figure}[H]
\includegraphics[height=12cm,width=1.0\linewidth]{data.png}
\caption{降维后数据集}
\end{figure}

\subsection{Kmeans方法聚类实现}
设置Kmeans方法模型所需聚成的类为3类,初始中心向量为随机选择,进行Kmeans聚类。
\begin{python}
KMean_model = KMeans(n_clusters=3, init='random', n_init=1).fit(X_rd)

y_predict = KMean_model.labels_  # 预测
clu_centers = KMean_model.cluster_centers_  # 聚类中心
\end{python}

\subsection{DBSCAN方法聚类实现}

设置DBSCAN的参数$\epsilon$ 单位为4,最小样本数为10,生成聚类模型。
\begin{python}
DBSCAN_model = DBSCAN(eps=4, min_samples=10, metric='euclidean').fit(X_rd)
\end{python}

\subsection{聚类模型性能评估}
使用教材中给出的聚类性能评估的外部指标和内部指标,对两种模型的聚类性能进行评估,具体代码如下:
\begin{python}
print('FMI‘s value:', fowlkes_mallows_score(y, y_predict))
print('RI’s value:', rand_score(y, y_predict))
print('DB‘s value', davies_bouldin_score(X_rd, y_predict))
\end{python}
\vfill
\section{模型实验结果对比分析}
经过Kmeans聚类后生成的聚类结果可视化后如下图:

\begin{figure}[H]
\includegraphics[height=12cm,width=1.0\linewidth]{Kmeans.png}
\caption{Kmeans聚类结果}
\end{figure}
\vfill
经DBSCAN方法聚类结果如下
\begin{figure}[H]
\includegraphics[height=12cm,width=1.0\linewidth]{DBSCAN.png}
\caption{DBSCAN聚类结果}
\end{figure}

从三维视图中可以看出kmeans方法能将原本样本设置的3个中心类别很好地区分出来,而DBSCAN方法则是将两个相近的簇合并认为是一个类别,总共生成了两个类。

计算分析比较二者的外部指标和内部指标如下表:

\begin{table}[H]
	\centering
	\begin{tabular}{c|ccc}
	\toprule[1.2pt]
	&FMI&RI&DB\\
	\midrule
	Kmeans&0.960679&0.973883&0.803419\\
	\midrule
	DBSCAN&0.693478&0.753170&2.382595\\
	\bottomrule[1.2pt]
	\end{tabular}
\end{table}

可以看到Kmeas的FMI与RI都非常高,DB指数小,聚类效果好;而DBSCAN的FMI,RI均偏低,而DB指数较高,聚类效果不如Kmeans方法。












\end{document}